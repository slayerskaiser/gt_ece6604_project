\documentclass[12pt,onecolumn]{IEEEtran}
\usepackage{mathtools}
\usepackage{bm}

\usepackage[nospace]{cite}
\bibliographystyle{IEEEtran}

\title{ECE 6604 Design Project:\\ Performance of STBC Schemes over Rayleigh Block-Fading Channel for 4G MIMO}
\author{Abhishek~Obla~Hema and~Klaus~Okkelberg}

\newcommand*\dif{\mathop{}\!\mathrm{d}}
\newcommand{\trans}{^\text{T}}
\newcommand{\herm}{^\text{H}}
\newcommand{\order}[1]{\mathcal{O}(#1)}
\DeclareMathOperator{\E}{E}
\DeclareMathOperator{\trace}{trace}

\begin{document}

\maketitle

\begin{abstract}
  Wireless mobile networks are a part of everyday life, but they have limited range and data rate to satisfy the demands of the exponentially increasing number of wireless mobile devices. In particular, high energy consumption and spectrum contraints mean that current 3.9G technology will not be able to satisfy the demands of mobile data traffic in the near future. To overcome these limitations, various diversity schemes are used, which fall into six categories, namely space, angle, polarization, frequency, and multipath. One implementation of a diversity scheme is to use MIMO links, which use multiple transmit and receive antennas to increase time and spatial diversity, the result of which is to increase range and data rate without requiring additional transmit power or bandwidth. To this end, this paper investigates the performance of space-time block coding (STBC) schemes for up to eight transmit antennas and up to two receive antennas, in line with current 4G standards for cellular phones. These schemes are analyzed for their spectral efficiency and bit error rates for transmitting data modulated using BPSK, QPSK, 16-QAM, and 64-QAM over a Rayleigh block-fading channel with additive white Gaussian noise. This is accomplished through software simulation in Matlab.
\end{abstract}

\section{Introduction}
%The current generation of wireless communication standards have to comply with ITU-R IMT-Advanced requirements to be considered 4G. These include 3GPP Long Term Evolution Advanced (LTE-A) and IEEE 802.16m WiMAX Release 2. However, these 4G standards are not widely deployed, and current 
%The fastest cellular technologies in current use are primarily 3.9G, including the 3GPP Long Term Evolution (LTE) and IEEE 802.16e mobile WiMAX standards. 

LTE, an abbreviation for Long-Term Evolution, commonly marketed as 4G LTE, is a standard for wireless communication of high-speed data for mobile phones and data terminals. It is based on the GSM/EDGE and UMTS/HSPA network technologies, increasing the capacity and speed using a different radio interface together with core network improvements. The standard is developed by the 3GPP (3rd Generation Partnership Project). LTE is the natural upgrade path for carriers with both GSM/UMTS networks and CDMA2000 networks. The different LTE frequencies and bands used in different countries will mean that only multi-band phones will be able to use LTE in all countries where it is supported.

     LTE stands for Long Term Evolution and is a registered trademark owned by ETSI (European Telecommunications Standards Institute) for the wireless data communications technology and a development of the GSM/UMTS standards. The goal of LTE was to increase the capacity and speed of wireless data networks using new DSP (digital signal processing) techniques and modulations that were developed around the turn of the millennium. A further goal was the redesign and simplification of the network architecture to an IP-based system with significantly reduced transfer latency compared to the 3G architecture. The LTE wireless interface is incompatible with 2G and 3G networks, so that it must be operated on a separate radio spectrum. The LTE specification provides downlink peak rates of 300 Mbit/s, uplink peak rates of 75 Mbit/s and QoS provisions permitting a transfer latency of less than 5ms in the radio access network. LTE has the ability to manage fast-moving mobiles and supports multi-cast and broadcast streams. LTE supports scalable carrier bandwidths, from 1.4 MHz to 20 MHz and supports both frequency division duplexing (FDD) and time-division duplexing (TDD). The IP-based network architecture, called the Evolved Packet Core (EPC) designed to replace the GPRS Core Network, supports seamless handovers for both voice and data to cell towers with older network technology such as GSM, UMTS and CDMA2000. The simpler architecture results in lower operating costs (for example, each E-UTRA cell will support up to four times the data and voice capacity supported by HSPA)

How is LTE configured for deployment?

LTE supports deployment on different frequency bandwidths. The current specification outlines the following bandwidth blocks: 1.4MHz, 3MHz, 5MHz, 10MHz, 15MHz, and 20MHz. Frequency bandwidth blocks are essentially the amount of space a network operator dedicates to a network. Depending on the type of LTE being deployed, these bandwidths have slightly different meaning in terms of capacity. That will be covered later, though. An operator may choose to deploy LTE in a smaller bandwidth and grow it to a larger one as it transitions subscribers off of its legacy networks (GSM, CDMA, etc.)

How LTE actually works?

LTE uses two different types of air interfaces (radio links), one for downlink (from tower to device), and one for uplink (from device to tower). By using different types of interfaces for the downlink and uplink, LTE utilizes the optimal way to do wireless connections both ways, which makes a better-optimized network and better battery life on LTE devices.

For the downlink, LTE uses an OFDMA (orthogonal frequency division multiple access) air interface as opposed to the CDMA (code division multiple access) and TDMA (time division multiple access) air interfaces . OFDMA (unlike CDMA and TDMA) mandates that MIMO (multiple in, multiple out) is used. Having MIMO means that devices have multiple connections to a single cell, which increases the stability of the connection and reduces latency tremendously. It also increases the total throughput of a connection. MIMO is what lets 802.11n WiFi reach speeds of up to 600Mbps, though most advertise up to 300-400Mbps. There is a significant disadvantage though. MIMO works better the further apart the individual carrier antennae are. On smaller phones, the noise caused by the antennae being so close to each other will cause LTE performance to drop. 

For the uplink (from device to tower), LTE uses the DFTS-OFDMA (discrete Fourier transform spread orthogonal frequency division multiple access) scheme of generating a SC-FDMA (single carrier frequency division multiple access) signal. As opposed to regular OFDMA, SC-FDMA is better for uplink because it has a better peak-to-average power ratio over OFDMA for uplink. LTE-enabled devices, in order to conserve battery life, typically don’t have a strong and powerful signal going back to the tower, so a lot of the benefits of normal OFDMA would be lost with a weak signal. Despite the name, SC-FDMA is still a MIMO system. LTE uses a SC-FDMA 1×2 configuration, which means that for every one antenna on the transmitting device, there are two antennae on the base station for receiving.

Discrete Fourier transform functions are often used to convert digital data into analog waveforms for decoding audio and video, but it can be used for outputting the proper radio frequencies too. However, LTE-Advanced uses higher order MIMO configurations for downlink and uplink.

The LTE technology itself also comes in two flavors: an FDD (frequency division duplex) variant and a TDD (time division duplex) variant. The most common variant being used is the FDD variant. The FDD variant uses separate frequencies for downlink and uplink in the form of a band pair. That means for every band that a phone supports, it actually uses two frequency ranges. These are known as paired frequency bands. For example, Verizon’s 10MHz network is in FDD, so the bandwidth is allocated for uplink and downlink. This is commonly noted as a 2x10MHz or 10+10 MHz configuration. Some also call it 10x10MHz, but this is mathematically incorrect, but they mean 10+10MHz. Some will also call it a 20MHz network, but this can be ambiguous. The TDD variant uses one single range of frequencies in a frequency band, but that band is segmented to support transmit and receive signals in a single frequency range.

For example, an LTE TDD network deployed on 20MHz of spectrum uses the whole chunk as one large block for frequency allocation purposes. For network bandwidth purposes, a LTE TDD network’s spectrum can be further divided to optimize for the type of network traffic (half up and half down, mostly down and a bit up, mostly up and a bit down, and so on).


4G Effects on Battery Life

By itself, LTE devices should last roughly as long as their HSPA+ equivalents because of the optimized radios for both downlink and uplink operations. The reason why LTE devices right now eat batteries for breakfast is because the network operators are forcing many of these devices into active dual-mode operation.

For Verizon Wireless, this means that most of their LTE devices connect to both CDMA2000 and LTE simultaneously and stay connected to both. This consumes twice the amount of battery for every minute we are connected than we would if we were connected only to CDMA2000 or LTE. Additionally, when we make calls on Verizon Wireless LTE phones, the CDMA2000 radio sucks down more power because you are talking. Sending and receiving text messages causes pulses of CDMA2000 activity, which cuts battery life more. Arguably, constantly changing radio states could be worse for battery life than a switch into one mode for a period of time and switching back, so text messages may actually kill the batteries faster.

Then there is handover. Handover is the operation in which a device switches from one network to another or from one tower to another. Handover is the critical component that makes any cellular wireless network possible. Without handover, a user would have to manually select a new tower every time the user leaves the range of a tower. For cellular networks, this is even more critical because the range of a tower is not very predictable due to factors outside of anyone’s control (like the weather, etc.). LTE supports handover like all other cellular wireless networks, but it improves on it by doing it much faster when handing over to a supported type of network or cell.

The paper is organized as follows. Section II provides background on diversity-increasing schemes used in current 3.9G and 4G standards. Section III discusses current STBC schemes. Section IV explains the simulation methodology. The simulation results are presented and analyzed in section V. Finally, section VI summarizes the results and discusses how they can be applied to future 5G technologies. The simulation code used for this paper is included in the appendix.

\section{Background}
% Section I: Background



Rayleigh fading of signals yields a very large performance loss by converting the exponential dependency of bit error probability $P_b$ on the mean received bit energy-to-noise ratio $E_b/N_0$ into an inverse linear one. Diversity is one remedy that improves the reliability of communication by providing the receiver with multiple independently-faded copies of the same information. These copies are often referred to as diversity branches. Diversity methods can be categorized as those that exploit (1) spatial, (2) angle, (3) polarization, (4) frequency, (5) multipath, and (6) time diversity. At the receiver, the diversity branches are combined together, with the most effective combining method depending, among other things, on the type of additive impairment. For additive white Gaussian noise (AWGN) dominant channels, maximal ratio combining (MRC) is optimal in the maximum likelihood (ML) sense. For co-channel interference (CCI) dominant channels, optimum combining performs better.

The performance of a diversity scheme is measured by its diversity gain, defined as the number of independent receptions of the same signal. The independent copies of the signal increase the signal-to-interference ratio and allow a reduction in transmission power without a performance loss. A MIMO system with $N_T$ transmit antennas and $N_R$ receive antennas has a maximum diversity gain equal to $N_TN_R$.


\section{Space-Time Block Coding}
% Section II: STBC

With the carrier frequencies used in current 3.9G and 4G standards, only up to two receive antennas can be used while maintaining spatial diversity. Thus, it is necessary to use transmitter diversity, which uses multiple transmit antennas to provide the receiver with multiple uncorrelated copies of the same signal. Simple forms of transmit diversity include selective transmit diversity for time division duplexed systems as well as time division transmit diversity, time-switched transmit diversity, and delay transmit diversity for frequency division duplexed systems. More complex forms of transmit diversity use space-time, space-frequency, or space-time-frequency coding.

For MIMO systems, knowledge of the channel state information (CSI) at the transmitter (CSIT) and at the receiver (CSIR) is necessary to choose a diversity technique~\cite{caire03,weingarten06}. In general, the channel can be estimated at the receiver using various techniques, so the knowledge of the CSIR can be assumed. At the transmitter, if the CSIT is known, then beamforming techniques are used to assure both the diversity gain and the array gain. However, without CSIT, space-time (ST) codes can still be used to assure the diversity gain. ST codes are a general class of error correcting codes in which the control symbols are inserted in both spatial and temporal domains.

The multi-layered space-time architecture was introduced by Foschini in~\cite{foschini96}. Later, Space-Time Trellis Codes (STTrC)~\cite{tarokh98} were proposed, which provide the optimal tradeoff between constellation size, data rate, diversity gain, and trellis complexity but have a greater decoding complexity. Regarding encoding and decoding complexity, Alamouti~\cite{alamouti98} introduced a simple repetition diversity scheme for two transmit antennas with maximum likelihood combining at the receiver. Alamouti's transmit diversity scheme provides a maximum diversity gain and no coding gain for a minimum decoding complexity. Later, Tarokh et al.\ generalized the Alamouti code for an arbitrary number of transmit antennas as the Space-Time Block Code (STBC)~\cite{tarokh99}.

The remainder of this section presents a mathematical model for a general MIMO system using STBC and details ML decoding methods for both linear and widely linear coding schemes. Additionally, some examples of STBCs are reviewed with emphasis on the code transmission matrix and the code parameters.

\subsection{Space-Time Block Coded MIMO System}
This paper considers digital wireless transmissions through a general MIMO system with $N_T$ transmit antennas and $N_R$ receive antennas, consisting of a space-time encoder, a Rayleigh block-fading channel with AWGN, and a space-time decoder with MRC and ML decoding. The channel is modeled as a discrete-time baseband-equivalent channel. This study assumes single-carrier modulated block transmission with zero padding is used to suppress intersymbol interference, see~\cite{wang00,wang04}.

\subsubsection{Space-Time Encoding}

\subsubsection{Space-Time Decoding}

\subsection{Examples of STBCs}

\subsubsection{Alamouti's STBC}
The Alamouti STBC uses two transmit antennas and any number of receive antennas.


\section{Methodology}
\input{sectionIII}

\section{Results}

\section{Summary}

\bibliography{report}

\appendix
\section{Alamouti Encoding}
\section{Alamouti Decoding}

%\begin{thebibliography}{99}
%
%\bibitem{gupta}
%  A. Gupta, H. S. Dhillon, S. Vishwanath, and J. G. Andrews, ``Downlink MIMO HetNets with Load Balancing'' \emph{IEEE Transactions on Communications}, Vol. 62, No. 11, pp. 4052-67, Nov. 2014.
%
%\bibitem{andrews}
%  J. G. Andrews, S. Buzzi, W. Choi, S. Hanly, A. Lozano, A. Soong, and J. C. Zhang, ``What will 5G be?'', \emph{IEEE Journal on Sel. Areas in Communications}, Vol. 32, No. 6, pp. 1065-82, June 2014.
%
%\bibitem{dhillon}
%  H. S. Dhillon, M. Kountouris and J. G. Andrews, ``Downlink MIMO HetNets: Modeling, Ordering Results and Performance Analysis'', \emph{IEEE Transactions on Wireless Communications}, vol. 12, no. 10, pp. 5208-5222, Oct. 2013.
%
%\bibitem{pappas}
%  N. Pappas and M. Kountouris, ``Performance Analysis of Distributed Cooperation under Uncoordinated Network Interference'', \emph{IEEE International Conference on Acoustics, Speech, and Signal Processing (ICASSP), 2014}, Florence, Italy, May 4-9, 2014.
%
%\bibitem{goonewardena}
%  M. Goonewardena, X. Jin, W. Ajib and H. Elbiaze, ``Competition vs. Cooperation: A Game-Theoretic Decision Analysis for MIMO HetNets'', \emph{IEEE International Conference on Communications, 2014}, Sydney, Australia, June 10-14, 2014.
%
%\bibitem{hoydis}
%  J. Hoydis, S. ten Brink, and M. Debbah, ``Massive MIMO in the UL/DL of cellular networks: How many antennas do we need?'' \emph{IEEE J. Sel. Areas Commun.}, vol. 31, no. 2, pp. 160–171, Feb. 2013.
%
%\bibitem{hoydis2}
%  J. Hoydis, C. Hoek, T. Wild, and S. ten Brink, ``Channel measurements for large antenna arrays,'' in \emph{Proc. IEEE International Symposium on Wireless Communication Systems}, Paris, France, Aug. 2012, pp. 811–815.
%
%\bibitem{bjornson}
%  E. Bj\"ornson, M. Kountouris, and M. Debbah, ``Massive MIMO and small cells: Improving energy efficiency by optimal soft-cell coordination,'' in \emph{International Conference on Telecommunications}, Casablanca, Morocco, May 2013.
%
%\bibitem{stuber}
%  St\"uber, G. L., Barry, J., McLaughlin, S. W., Li, Y., Ingram, M.A., and Pratt, T. G., ``Broadband MIMO-OFDM Wireless Communications,'' \emph{IEEE Proceedings}, pp. 271-294, April 2004.
%
%\bibitem{geirhofer}
%  S. Geirhofer and P. Gaal, ``Coordinated Multi-Point Transmission in 3GPP LTE Heterogeneous Networks,'' \emph{Proc. IEEE GLOBECOM Wksp. Emerging Technologies for LTEAdvanced and Beyond-4G}, Anaheim, CA, Dec. 2012.
%
%\bibitem{bhushan}
%  N. Bhushan, Junyi Li, D. Malladi, R. Gilmore, D. Brenner, A. Damnjanovic, R. Sukhavasi, C. Patel, S. Geirhofer, ``Network densification: the dominant theme for wireless evolution into 5G,'' in \emph{Communications Magazine, IEEE}, vol.52, no.2, pp.82-89, February 2014.
%
%\bibitem{Obaidat}
%  \emph{Modeling and Simulation of Computer Networks and Systems: Methodologies and Applications}, M. S. Obaidat, F. Zarai, and P. Nicopolitidis, Ed., Waltham, MA: Elsevier Science, 2015.
%
%\end{thebibliography}

\end{document}
