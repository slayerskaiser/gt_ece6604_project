\documentclass[12pt,onecolumn]{IEEEtran}

\title{ECE 6604 Design Project:\\ Performance of STBC Schemes over Rayleigh Block-Fading Channel for 4G MIMO}
\author{Abhishek~Obla~Hema and~Klaus~Okkelberg}

\begin{document}

\maketitle

\begin{abstract}
  Wireless mobile networks are a part of everyday life, but they have limited range and data rate to satisfy the demands of the exponentially increasing number of wireless mobile devices. In particular, high energy consumption and spectrum contraints mean that current 3.9G technology will not be able to satisfy the demands of mobile data traffic in the near future. To overcome these limitations, various diversity schemes are used, which fall into six categories, namely space, angle, polarization, frequency, and multipath. One implementation of a diversity scheme is to use MIMO links, which use multiple transmit and receive antennas to increase time and spatial diversity, the result of which is to increase range and data rate without requiring additional transmit power or bandwidth. To this end, this paper investigates the performance of space-time block coding (STBC) schemes for up to eight transmit antennas and up to two receive antennas, in line with current 4G standards for cellular phones. These schemes are analyzed for their spectral efficiency and bit error rates for transmitting data modulated using BPSK, QPSK, 16-QAM, and 64-QAM over a Rayleigh block-fading channel with additive white Gaussian noise. This is accomplished through software simulation in Matlab.
\end{abstract}

\section{Introduction}
%The current generation of wireless communication standards have to comply with ITU-R IMT-Advanced requirements to be considered 4G. These include 3GPP Long Term Evolution Advanced (LTE-A) and IEEE 802.16m WiMAX Release 2. However, these 4G standards are not widely deployed, and current 
%The fastest cellular technologies in current use are primarily 3.9G, including the 3GPP Long Term Evolution (LTE) and IEEE 802.16e mobile WiMAX standards. 

The paper is organized as follows. Section II provides background on diversity-increasing schemes used in current 3.9G and 4G standards. Section III discusses current STBC schemes. Section IV explains the simulation methodology. The simulation results are presented and analyzed in section V. Finally, section VI summarizes the results and discusses how they can be applied to future 5G technologies. The simulation code used for this paper is included in the appendix.

\section{Background}
% Section I: Background



Rayleigh fading of signals yields a very large performance loss by converting the exponential dependency of bit error probability $P_b$ on the mean received bit energy-to-noise ratio $E_b/N_0$ into an inverse linear one. Diversity is one remedy that improves the reliability of communication by providing the receiver with multiple independently-faded copies of the same information. These copies are often referred to as diversity branches. Diversity methods can be categorized as those that exploit (1) spatial, (2) angle, (3) polarization, (4) frequency, (5) multipath, and (6) time diversity. At the receiver, the diversity branches are combined together, with the most effective combining method depending, among other things, on the type of additive impairment. For additive white Gaussian noise (AWGN) dominant channels, maximal ratio combining (MRC) is optimal in the maximum likelihood (ML) sense. For co-channel interference (CCI) dominant channels, optimum combining performs better.

The performance of a diversity scheme is measured by its diversity gain, defined as the number of independent receptions of the same signal. The independent copies of the signal increase the signal-to-interference ratio and allow a reduction in transmission power without a performance loss. A MIMO system with $N_T$ transmit antennas and $N_R$ receive antennas has a maximum diversity gain equal to $N_TN_R$.


\section{Space-Time Block Coding}
% Section II: STBC

With the carrier frequencies used in current 3.9G and 4G standards, only up to two receive antennas can be used while maintaining spatial diversity. Thus, it is necessary to use transmitter diversity, which uses multiple transmit antennas to provide the receiver with multiple uncorrelated copies of the same signal. Simple forms of transmit diversity include selective transmit diversity for time division duplexed systems as well as time division transmit diversity, time-switched transmit diversity, and delay transmit diversity for frequency division duplexed systems. More complex forms of transmit diversity use space-time, space-frequency, or space-time-frequency coding.

For MIMO systems, knowledge of the channel state information (CSI) at the transmitter (CSIT) and at the receiver (CSIR) is necessary to choose a diversity technique~\cite{caire03,weingarten06}. In general, the channel can be estimated at the receiver using various techniques, so the knowledge of the CSIR can be assumed. At the transmitter, if the CSIT is known, then beamforming techniques are used to assure both the diversity gain and the array gain. However, without CSIT, space-time (ST) codes can still be used to assure the diversity gain. ST codes are a general class of error correcting codes in which the control symbols are inserted in both spatial and temporal domains.

The multi-layered space-time architecture was introduced by Foschini in~\cite{foschini96}. Later, Space-Time Trellis Codes (STTrC)~\cite{tarokh98} were proposed, which provide the optimal tradeoff between constellation size, data rate, diversity gain, and trellis complexity but have a greater decoding complexity. Regarding encoding and decoding complexity, Alamouti~\cite{alamouti98} introduced a simple repetition diversity scheme for two transmit antennas with maximum likelihood combining at the receiver. Alamouti's transmit diversity scheme provides a maximum diversity gain and no coding gain for a minimum decoding complexity. Later, Tarokh et al.\ generalized the Alamouti code for an arbitrary number of transmit antennas as the Space-Time Block Code (STBC)~\cite{tarokh99}.

The remainder of this section presents a mathematical model for a general MIMO system using STBC and details ML decoding methods for both linear and widely linear coding schemes. Additionally, some examples of STBCs are reviewed with emphasis on the code transmission matrix and the code parameters.

\subsection{Space-Time Block Coded MIMO System}
This paper considers digital wireless transmissions through a general MIMO system with $N_T$ transmit antennas and $N_R$ receive antennas, consisting of a space-time encoder, a Rayleigh block-fading channel with AWGN, and a space-time decoder with MRC and ML decoding. The channel is modeled as a discrete-time baseband-equivalent channel. This study assumes single-carrier modulated block transmission with zero padding is used to suppress intersymbol interference, see~\cite{wang00,wang04}.

\subsubsection{Space-Time Encoding}

\subsubsection{Space-Time Decoding}

\subsection{Examples of STBCs}

\subsubsection{Alamouti's STBC}
The Alamouti STBC uses two transmit antennas and any number of receive antennas.


\section{Methodology}

\section{Results}

\section{Summary}

\bibliographystyle{ieeetr}
\bibliography{report}

\appendix
\section{Alamouti Encoding}
\section{Alamouti Decoding}

%\begin{thebibliography}{99}
%
%\bibitem{gupta}
%  A. Gupta, H. S. Dhillon, S. Vishwanath, and J. G. Andrews, ``Downlink MIMO HetNets with Load Balancing'' \emph{IEEE Transactions on Communications}, Vol. 62, No. 11, pp. 4052-67, Nov. 2014.
%
%\bibitem{andrews}
%  J. G. Andrews, S. Buzzi, W. Choi, S. Hanly, A. Lozano, A. Soong, and J. C. Zhang, ``What will 5G be?'', \emph{IEEE Journal on Sel. Areas in Communications}, Vol. 32, No. 6, pp. 1065-82, June 2014.
%
%\bibitem{dhillon}
%  H. S. Dhillon, M. Kountouris and J. G. Andrews, ``Downlink MIMO HetNets: Modeling, Ordering Results and Performance Analysis'', \emph{IEEE Transactions on Wireless Communications}, vol. 12, no. 10, pp. 5208-5222, Oct. 2013.
%
%\bibitem{pappas}
%  N. Pappas and M. Kountouris, ``Performance Analysis of Distributed Cooperation under Uncoordinated Network Interference'', \emph{IEEE International Conference on Acoustics, Speech, and Signal Processing (ICASSP), 2014}, Florence, Italy, May 4-9, 2014.
%
%\bibitem{goonewardena}
%  M. Goonewardena, X. Jin, W. Ajib and H. Elbiaze, ``Competition vs. Cooperation: A Game-Theoretic Decision Analysis for MIMO HetNets'', \emph{IEEE International Conference on Communications, 2014}, Sydney, Australia, June 10-14, 2014.
%
%\bibitem{hoydis}
%  J. Hoydis, S. ten Brink, and M. Debbah, ``Massive MIMO in the UL/DL of cellular networks: How many antennas do we need?'' \emph{IEEE J. Sel. Areas Commun.}, vol. 31, no. 2, pp. 160–171, Feb. 2013.
%
%\bibitem{hoydis2}
%  J. Hoydis, C. Hoek, T. Wild, and S. ten Brink, ``Channel measurements for large antenna arrays,'' in \emph{Proc. IEEE International Symposium on Wireless Communication Systems}, Paris, France, Aug. 2012, pp. 811–815.
%
%\bibitem{bjornson}
%  E. Bj\"ornson, M. Kountouris, and M. Debbah, ``Massive MIMO and small cells: Improving energy efficiency by optimal soft-cell coordination,'' in \emph{International Conference on Telecommunications}, Casablanca, Morocco, May 2013.
%
%\bibitem{stuber}
%  St\"uber, G. L., Barry, J., McLaughlin, S. W., Li, Y., Ingram, M.A., and Pratt, T. G., ``Broadband MIMO-OFDM Wireless Communications,'' \emph{IEEE Proceedings}, pp. 271-294, April 2004.
%
%\bibitem{geirhofer}
%  S. Geirhofer and P. Gaal, ``Coordinated Multi-Point Transmission in 3GPP LTE Heterogeneous Networks,'' \emph{Proc. IEEE GLOBECOM Wksp. Emerging Technologies for LTEAdvanced and Beyond-4G}, Anaheim, CA, Dec. 2012.
%
%\bibitem{bhushan}
%  N. Bhushan, Junyi Li, D. Malladi, R. Gilmore, D. Brenner, A. Damnjanovic, R. Sukhavasi, C. Patel, S. Geirhofer, ``Network densification: the dominant theme for wireless evolution into 5G,'' in \emph{Communications Magazine, IEEE}, vol.52, no.2, pp.82-89, February 2014.
%
%\bibitem{Obaidat}
%  \emph{Modeling and Simulation of Computer Networks and Systems: Methodologies and Applications}, M. S. Obaidat, F. Zarai, and P. Nicopolitidis, Ed., Waltham, MA: Elsevier Science, 2015.
%
%\end{thebibliography}

\end{document}
