%%% Local Variables:
%%% mode: latex
%%% TeX-master: "report"
%%% End:

% Section IV: Methodology

This study investigated the performance of eight STBC codes for a STBC-MIMO system transmitting over a Rayleigh flat block-fading channel with AWGN using software simulation. The eight MIMO cases of $1\times1$, $1\times2$ with MRC, $2\times1$ and $2\times2$ using Alamouti's $C(2,2,2)$ code, $4\times2$ using $C(4,2,4)$, $C(4,3,4)$ and $C(4,4,4)$, and $8\times2$ using $C(8,8,16)$ are considered for the four modulation methods of BPSK, QPSK, 16-QAM, and 64-QAM. The MIMO and modulation choices represent realistic possibilities for current 4G systems. BPSK, while not in current 4G standards, was included for better comparison between half-rate and full-rate codes.

The channel was modeled as a discrete-time baseband-equivalent channel, and it was assumed that there was no ISI and ICI. The channel coherence time and time slot period were chosen to be $T_c=T=1/(15~\text{kHz})$ to represent flat fading, where 15~kHz is the subcarrier spacing for LTE. In addition, block fading was assumed, with the channel quasi-static over $4T$ time slots.

At the transmitter, the modulated symbols were scaled to have an average energy of unity. For simplicity, OFDM was not used, because in theory IFFT at the transmitter and FFT at the receiver have no effect on the performance of a linear system according to Parseval's theorem. Then, the modulated symbols are transmitted over a Rayleigh faded channel with Doppler frequencies corresponding to speeds of $v={}$0, 5, 15, and 45~m/s for a carrier frequency of $f_c=1800$~MHz. For the stationary case, the uncorrelated path gains were directly generated. For the moving cases, they were generated according to the statistical method for multiple uncorrelated faded envelopes given by~\cite{stuber12}. In addition, white Gaussian noise is added the output of the Rayleigh channel, with the energy of the noise scaled to give the desired $E_b/N_0$.

It is assumed that CSIR is known, so at the receiver, the pseudo-inverse of the equivalent channel matrix $\bm{H}_{ef}^\dagger$ is used to perform the MRC. In the case of the widely linear code $C(4,3,4)$, additional steps are needed to generate the augmented output matrix and extract the desired symbols from the augmented input matrix. This is followed by ML demodulation of the combined symbols.

To evaluate the performance of the STBC-MIMO system, the bit error rate (BER) $P_b$ is measured and plotted versus the bit energy-to-noise ratio $E_b/N_0$. The simulation was performed using packets of 120 symbols and $10^5$ total packets were used per data point. The range of $E_b/N_0$ values considered was from 0 to 20~dB with a spacing of 2~dB.

The simulation results were used to generate numerous plots, each studying a different performance aspect of STBCs. First, the diversity gains of the uncoded SISO channel, the Alamouti coded channels, and the uncoded MRC channel were compared for a stationary receiver to verify that the simulation achieves the expected diversity gain and to analyze the performance of uncoded performance versus coded performance. Second, the performance of codes with $N_T=4$ and a stationary receiver were compared. This results shows the tradeoff between spectral efficiency and diversity gain. Third, the best performing codes for each number of transmit antennas are compared for a stationary receiver to study how the diversity gain affects performance over the range bit energy-to-noise ratios. Fourth, the first and second set of results described perviously are redone for a moving receiver in order to compare the effect of orthogonality and linearness of the code to its performance in deep fading situations. Lastly, the third set of results are redone for a moving receiver to compare how the performance of the best performaning codes while stationary changes for a moving receiver. The last two sets of results are only displayed for QPSK due to the close relationship between all the modulation methods. Additionally, for display reasons, the results of BPSK are only included for the case of $N_T=4$ where comparison of the spectral efficiencies of the codes was analyzed.

The simulations were done in a mixture of Matlab versions 2012b, 2014a, 2014b, and 2015a. Care was taken to ensure results were comparable by using the same seed to the random number generator for each run to ensure repeatability in any Matlab version. The modulation and demodulation as well as the AWGN used built-in Matlab functions, while the STBC encoding and decoding and the generation of the Rayleigh fading channel for the various Doppler frequencies used code written by the authors. These Matlab functions are included in the appendix of this paper along with a sample script showing how to use the functions to duplicate the simulation results.