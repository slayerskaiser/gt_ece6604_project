% Section I: Background



Rayleigh fading of signals yields a very large performance loss by converting the exponential dependency of bit error probability $P_b$ on the mean received bit energy-to-noise ratio $E_b/N_0$ into an inverse linear one. Diversity is one remedy that improves the reliability of communication by providing the receiver with multiple independently-faded copies of the same information. These copies are often referred to as diversity branches. Diversity methods can be categorized as those that exploit (1) spatial, (2) angle, (3) polarization, (4) frequency, (5) multipath, and (6) time diversity. At the receiver, the diversity branches are combined together, with the most effective combining method depending, among other things, on the type of additive impairment. For additive white Gaussian noise (AWGN) dominant channels, maximal ratio combining (MRC) is optimal in the maximum likelihood (ML) sense. For co-channel interference (CCI) dominant channels, optimum combining performs better.

The performance of a diversity scheme is measured by its diversity gain, defined as the number of independent receptions of the same signal. The independent copies of the signal increase the signal-to-interference ratio and allow a reduction in transmission power without a performance loss. A MIMO system with $N_T$ transmit antennas and $N_R$ receive antennas has a maximum diversity gain equal to $N_TN_R$.
