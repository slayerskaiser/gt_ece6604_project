%%% Local Variables:
%%% mode: latex
%%% TeX-master: "report"
%%% End:


% Section II: STBC

With the carrier frequencies used in current 3.9G and 4G standards, only up to two receive antennas can be used while maintaining spatial diversity. Thus, it is necessary to use transmitter diversity, which uses multiple transmit antennas to provide the receiver with multiple uncorrelated copies of the same signal. Simple forms of transmit diversity include selective transmit diversity for time division duplexed systems as well as time division transmit diversity, time-switched transmit diversity, and delay transmit diversity for frequency division duplexed systems. More complex forms of transmit diversity use space-time, space-frequency, or space-time-frequency coding.

For MIMO systems, knowledge of the channel state information (CSI) at the transmitter (CSIT) and at the receiver (CSIR) is necessary to choose a diversity technique~\cite{caire03,weingarten06}. In general, the channel can be estimated at the receiver using various techniques (see~\cite{nasseri10,bjornson10,biguesh06}), so the knowledge of the CSIR can be assumed. At the transmitter, if the CSIT is known, then beamforming techniques are used to assure both the diversity gain and the array gain. However, without CSIT, space-time (ST) codes can still be used to assure the diversity gain. ST codes are a general class of error correcting codes in which the control symbols are inserted in both spatial and temporal domains.

The multi-layered space-time architecture was introduced by Foschini in~\cite{foschini96}. Later, Space-Time Trellis Codes (STTrC)~\cite{tarokh98} were proposed, which provide the optimal tradeoff between constellation size, data rate, diversity gain, and trellis complexity but have a greater decoding complexity. Regarding encoding and decoding complexity, Alamouti~\cite{alamouti98} introduced a simple repetition diversity scheme for two transmit antennas with maximum likelihood combining at the receiver. Alamouti's transmit diversity scheme provides a maximum diversity gain and no coding gain for a minimum decoding complexity. Later, Tarokh et al.\ generalized the Alamouti code for an arbitrary number of transmit antennas as the Space-Time Block Code (STBC)~\cite{tarokh99}.

The remainder of this section presents a mathematical model for a general MIMO system using STBC and details ML decoding methods for both linear and widely linear coding schemes. Additionally, some examples of STBCs are reviewed with emphasis on the code transmission matrix and the code parameters.

\subsection{Space-Time Block Coded MIMO System}
This paper considers digital wireless transmissions through a general MIMO system with $N_T$ transmit antennas and $N_R$ receive antennas, consisting of a ST encoder, a Rayleigh block-fading channel with AWGN, and a ST decoder with MRC and ML decoding. The channel is modeled as a discrete-time baseband-equivalent channel. This study assumes single-carrier modulated block transmission with zero padding is used to suppress intersymbol interference, as in~\cite{wang00,wang04} and uncorrelated path gains for the receive antennas.

\subsubsection{Space-Time Encoding}
Assume that the input to the ST encoder is a $k$-dimensional vector of complex symbols $\bm{s}=[s_1\,s_2\,\cdots\,s_k]\trans$. The encoding method for the STBC is defined by its transmission matrix in the form 
\[ \bm{G} = \begin{bmatrix}
  g_{11} & g_{12} & \cdots & g_{1N_T} \\
  g_{21} & g_{22} & \cdots & g_{2N_T} \\
  \vdots & \vdots &       & \vdots \\
  g_{T1} & g_{T2} & \cdots & g_{TN_T}
\end{bmatrix}, \]
where $g_{ij}$ is the modulated symbol transmitted in the $i$th time slot from the $j$th antenna. Note that the $g_{ij}$ are combinations of the input complex symbols and their conjugates and the encoding is independent of the number of receive antennas $N_R$.

During each time slot $i=1,\dots,T$, the symbols are sent simultaneously from the $N_T$ transmit antennas. Because $k$ symbols are transmitted during $T$ time slots, the code rate is $R=k/T$, where the maximum possible rate is unity~\cite{tarokh99}. For simplicity, it is useful to refer to a code by its parameters as $C(N_T,k,T)$. With this notation, an uncoded transmission is $C(N_T,1,1)$ and Alamouti code with two transmit antennas and full rate is $C(2,2,2)$.

In practice, the class of useful codes is restricted to Orthogonal STBCs (OSTBCs), for which the transmission matrix $\bm{G}$ satisfies~\cite{alamouti98}:
\begin{itemize}
\item \emph{linearity}: its elements $g_{ij}$ are linear combinations of the input symbols and their conjugates
\item \emph{orthogonality}:
  \[ \bm{G}\herm\bm{G} = \Vert\bm{s}\Vert_2^2 \mathbf{I}_{N_T} = \left(\sum_{i=1}^k |s_i|^2\right) \mathbf{I}_{N_T} , \]
\end{itemize}
where $\Vert\cdot\Vert_2$ is the Euclidean norm and $\mathbf{I}_{N_T}$ is the $N_T\times N_T$ identity matrix. OSTBCs have the advantage of providing full diversity equal to $N_TN_R$ as well as making MRC of the received signals equivalent to the linear minimum mean square error (MMSE) estimate, as will be detailed in the decoding section. Note that these codes have very little or no coding gain in contrast to space-time trellis codes that spread a conventional trellis code over space and time.

Orthogonal matrices can be designed using Hurwitz-Radon (HR) theory~\cite{seberry05,wolfe76}, which shows that a HR family with $N-1$ matrices of size $N\times N$ exists only if $N\in\{2,4,8\}$. STBC transmission matrices can be constructed from the HR family of matrices. For real input constellations, square STBC matrices with full rate of unity exist only for $N_T\in\{2,4,8\}$. However, non-square full rate STBC matrices can be designed for $N_T\in\{3,5,6,7\}$ by removing columns from the square full rate matrices.

For complex input constellations, a full rate OSTBC only exists for $N_T=2$. However, for $N_T>2$, $R=1/2$ codes can be constructed for complex constellations from full rate codes for real constellations~\cite{tarokh99}, in essence sacrificing code rate to gain an orthogonal design. In addition, $R=3/4$ codes exist for $N_T=4$ but the authors do not know of any orthogonal codes with $R>1/2$ for $N_T>4$. Also of interest are Quasi-Orthogonal STBCs (QOSTBCs), in which only adjacent columns of the transmission matrix are orthogonal. QOSTBCs can achieve full rate communication at the expense of decoding complexity and diversity gain due to the interference~\cite{jafarkhani05}.

\subsubsection{Space-Time Coded MIMO Channel}
The transmission of the ST coded symbols over the MIMO channel considered by this paper is subject to fading and additive noise. Let $h_{ijk}$ denote the path gain between in the $i$th time slot between the $j$th transmit antenna and the $k$th receive antenna. Then, the received symbol $y_{ik}$ in the $i$th time slot by the $k$th receive antenna is
\[ y_{ik} = [h_{i,1,k}\,h_{i,2,k}\,\cdots\,h_{i,N_T,k}] \begin{bmatrix} x_{i,1} \\ x_{i,2} \\ \vdots \\ x_{i,N_T} \end{bmatrix} + n_{ik}, \]
where $x_{ij}$ is the symbol transmitted by the $j$th transmit antenna in the $i$th time slot and $n_{ik}$ is the additive noise for the $k$th receive antenna at the $i$th time slot. Note that the $x_{ij}$ are combinations, not necessarily linear, of the input symbols $s_i$ and their conjugates, so this system is difficult to solve in general.

For a block-fading channel with coherence time $T_c$, it is assumed that the fading coefficients $h_{ijk}$ remain constant within a frame of length $T_c$ time slots. With Rayleigh fading, these coefficients are generated as independent complex Gaussian random variables with zero mean and variance $\Omega_p/2$ in both the real and imaginary dimensions, where $\Omega_p$ is the average energy of the transmitted discrete-time symbol. In addition, the additive noise is assumed to be complex Gaussian random variables with zero mean and variance $1/(2E_b/N_0)$, where the $E_b/N_0$ is the bit energy-to-noise ratio of the channel.

\subsubsection{Space-Time Decoding}
ST decoding for a STBC consists of MRC followed by ML decoding of the combined symbols. For an OSTBC, MRC is equivalent to solving a linear system, which is advantageous for decoding performance. As the combined symbols are independent and assuming no ISI, the ML decoding can be performed independently of the MRC. A similar approach works for QOSTBC, but the performance is reduced to due interference between symbols.

For a general OSTBC or QOSTBC, the effect of the channel can be written in matrix form as
\[ \bar{\bm{y}} = [\bm{H}_1\,\bm{H}_2] \begin{bmatrix*}[l] \bm{x} \\ \bm{x}^\ast\end{bmatrix*} + \bar{\bm{n}}, \]
where $\bm{x}$ is the input to the STBC encoder, $\bar{\bm{y}}$ is the output of the MIMO channel, $\bar{\bm{n}}$ is the AWGN, and $\bm{H}_1$ and $\bm{H}_2$ are matrices representing the equivalent channel formed by the ST encoder and the MIMO channel. For a linearly decodable ST code where $\bm{x}$ and its complex conjugate $\bm{x}^\ast$ are uncorrelated, it is possible to rewrite the equation as
\[ \begin{bmatrix} \bar{\bm{y}}_1 \\ \bar{\bm{y}}_2^\ast \end{bmatrix} = \begin{bmatrix} \bm{H}_1 \\ \bm{H}_2^\ast \end{bmatrix} \bm{x} + \begin{bmatrix} \bm{n}_1 \\ \bm{n}_2^\ast \end{bmatrix}, \]
where $\bar{\bm{y}}=[\bar{\bm{y}}_1\trans\,\bar{\bm{y}}_2\trans]\trans$ and $\bar{\bm{n}}=[\bar{\bm{n}}_1\trans\,\bar{\bm{n}}_2\trans]\trans$. Then, let $\bm{y}=[\bar{\bm{y}}_1\trans\,\bar{\bm{y}}_2\trans]\trans$, $\bm{H}_{ef}=[\bm{H}_1\trans\,\bm{H}_2\herm]\trans$, and $\bm{n}=[\bar{\bm{n}}_1\trans\,\bar{\bm{n}}_2\trans]\trans$ to obtain:
\begin{equation}
  \bm{y} = \bm{H}_{ef} \bm{x} + \bm{n},
  \label{eq:channel_output}
\end{equation}
where $\bm{H}_{ef}$ is the matrix of the equivalent channel formed by the ST encoder and the MIMO channel. Note that for an OSTBC, $\bm{H}_{ef}$ is an orthogonal matrix over all channel realizations because
\[ \bm{H}_{ef}\herm\bm{H}_{ef} = \Vert\bm{H}\Vert_F^2 \mathbf{I}_k, \]
where $\bm{H}$ is the channel matrix, $\Vert\cdot\Vert_F$ is the Frobenius norm, and $k$ is the dimension of the input.

Assuming perfect CSIR, the MRC coefficients equal the complex conjugate of the equivalent channel matrix~\cite{larsson03,jafarkhani05}:
\begin{equation}
  \tilde{\bm{x}} = \bm{H}_{ef}\herm \bm{y}, \label{eq:mrc}
\end{equation}
where $\tilde{\bm{x}}$ is the vector of the combined symbols.

Using \eqref{eq:channel_output} and assuming an orthogonal ST code, \eqref{eq:mrc} can be written as
\[ \tilde{\bm{x}} = \bm{H}_{ef}\herm\bm{H}_{ef}\bm{x} + \bm{H}_{ef}\herm\bm{n} = \Vert\bm{H}\Vert_F^2 \mathbf{I}_k \bm{x} + \tilde{\bm{n}}, \]
where $\tilde{\bm{n}}$ has zero-mean and autocorrelation $\E[\tilde{\bm{n}}\tilde{\bm{n}}\herm] = \sigma_n^2\Vert\bm{H}\Vert_F^2\mathbf{I}_k$. Thus, the combined symbols are decoupled into
\begin{equation}
  \tilde{x}_i = \Vert\bm{H}\Vert_F^2 x_i + \tilde{n}_i, \quad i=1,\dots,k,
  \label{eq:mrc_decoup}
\end{equation}
where $k$ is the dimension of the $\bm{x}$. With this decoupling, ML decoding is trivial for the case of AWGN.

To see why MRC for the case of linearly decodable orthogonal ST codes is equivalent to the linear MMSE solution, consider solving for the input $\bm{x}$ in \eqref{eq:channel_output} using the pseudo-inverse:
\[ \hat{\bm{x}} = \bm{H}_{ef}^\dagger \bm{y} = (\bm{H}_{ef}\herm\bm{H}_{ef})^{-1}\bm{H}_{ef}\herm \bm{y} = \frac{1}{\Vert\bm{H}\Vert_F^2} \mathbf{I}_k \tilde{\bm{x}} = \bm{x} + \frac{1}{\Vert\bm{H}\Vert_F^2} \tilde{n}, \]
which is equivalent to the solution for \eqref{eq:mrc_decoup}.

For a nonlinearly decodable ST code where $\bm{x}$ and its complex conjugate $\bm{x}^\ast$ are correlated, it is still possible to perform MRC and ML decoding by solving a linear system of augmented symbols given by
\[ \begin{bmatrix} \bar{\bm{y}} \\ \bar{\bm{y}}^\ast \end{bmatrix}
= \begin{bmatrix} \bm{H}_1 & \bm{H}_2 \\ \bm{H}_2^\ast & \bm{H}_1^\ast \end{bmatrix}
\begin{bmatrix*}[l] \bm{x} \\ \bm{x}^\ast\end{bmatrix*} + \begin{bmatrix} \bar{\bm{n}} \\ \bar{\bm{n}}^\ast \end{bmatrix} . \]
This type of system is refered to as a widely linear system~\cite{dama13,picinbono95}. Then, the MMSE solution to this system is given by the pseudo-inverse as before and the ML decoding of the input symbols is given by the first half of the augmented input vector.

For QOSTBC, while the decoding is more complex than the diagonal decoding presented in \eqref{eq:mrc_decoup}, the system is still linear so a linear MMSE solution still exists. This increased decoding complexity is offset by its increased data rate over OSTBC, particularly for low $E_b/N_0$.

\subsection{Examples of STBCs}
The examples of STBCs given in this section are all for complex input constellations. These codes were chosen to give a good representation of current orthogonal or quasi-orthgonal codes, which have the advantage of being easily decodable by mobile devices using a linear MMSE solution. These examples only include $N_T\in\{2,4,8\}$, because current 4G standards only define up to eight transmit antennas and codes for other numbers of transmit antennas can be obtained by eliminating one or more columns from the transmission matrix of a code with greater $N_T$.

\subsubsection{Alamouti's STBC}
The Alamouti STBC uses two transmit antennas and is described by the transmission matrix
\[ \bm{G} = \begin{bmatrix} s_1 & s_2 \\ -s_2^\ast & s_1^\ast \end{bmatrix} . \]
This code uses two time slots to transmit two symbols over two antennas, making this a $C(2,2,2)$. The two antennas are assumed to have equal transmit power of $\Omega_p/2$ since CSIT is unknown and so the sum of their powers equals the transmit power of a singular transmit antenna.

For one receive antenna and assuming block-fading, the received symbols are
\begin{gather*}
  y_1 = h_1 s_1 + h_2 s_2 + n_1 \\
  y_2 = -h_1 s_2^\ast + h_2 s_1^\ast + n_2.
\end{gather*}
Taking the complex conjugate of the second equation yields
\begin{gather*}
  y_1 = h_1 s_1 + h_2 s_2 + n_1 \\
  y_2^\ast = h_2^\ast s_1 - h_1^\ast s_2 + n_2^\ast,
\end{gather*}
or in matrix form,
\[ \begin{bmatrix} y_1 \\ y_2^\ast \end{bmatrix} =
\begin{bmatrix} h_1 & h_2 \\ h_2^\ast & -h_1^\ast \end{bmatrix}
\begin{bmatrix} s_1 \\ s_2 \end{bmatrix} + \begin{bmatrix} n_1 \\ n_2^\ast \end{bmatrix}. \]
The MRC output of this system is
\begin{gather*}
  \tilde{s}_1 = h_1^\ast y_1 + h_2 y_2^\ast \\
  \tilde{s}_2 = h_2^\ast y_1 - h_1 y_2^\ast,
\end{gather*}
which can also be written as
\begin{gather*}
  \hat{s}_1 = (|h_1|^2+|h_2|^2) s_1 + h_1^\ast n_1 + h_2 n_2^\ast \\
  \hat{s}_2 = (|h_1|^2+|h_2|^2) s_2 - h_1 n_2^\ast + h_2^\ast n_1.
\end{gather*}
The combined symbols have the expected form given by \eqref{eq:mrc_decoup}. It can be seen that the Alamouti's $2\times1$ transmit diversity scheme is equivalent to a $1\times2$ receive divesity scheme with MRC. This result can be extended to any number of receive antennas, where Alamouti's $2\times N_R$ transmit diversity scheme is equivalent to a $1\times 2N_R$ diversity with MRC~\cite{stuber12}.

\subsubsection{$C(4,4,8)$}
For $N_T=4$, Tarokh et al.\ proposed a $R=1/2$ code~\cite{tarokh99}:
\[ \bm{G} = \begin{bmatrix}
  s_1 & s_2 & s_3 & s_4 \\
  -s_2 & s_1 & -s_4 & s_3 \\
  -s_3 & s_4 & s_1 & -s_2 \\
  -s_4 & -s_3 & s_2 & s_1 \\
  s_1^\ast & s_2^\ast & s_3^\ast & s_4^\ast \\
  -s_2^\ast & s_1^\ast & -s_4^\ast & s_3^\ast \\
  -s_3^\ast & s_4^\ast & s_1^\ast & -s_2^\ast \\
  -s_4^\ast & -s_3^\ast & s_2^\ast & s_1^\ast
\end{bmatrix}. \]
It can be see that the symbols transmitted during the second set of four time slots are the complex conjugates of those transmitted during the first set. In fact, a transmission matrix made up of the first four rows of this one represents a full-rate code for real constellations. This demonstrates how a $R=1/2$ code for complex constellations can be generated from any full-rate code for real constellations.

\subsubsection{$C(4,3,4)$}
To increase the bandwidth efficiency, Tarokh et al.\ also proposed a $R=3/4$ code~\cite{tarokh99}:
\[ \bm{G} = \begin{bmatrix}
  s_1 & s_2 & \frac{s_3}{\sqrt{2}} & \frac{s_3}{\sqrt{2}} \\
  -s_2^\ast s_1^\ast & \frac{s_3}{\sqrt{2}} & -\frac{s_3}{\sqrt{2}} \\
  \frac{s_3^\ast}{\sqrt{2}} & \frac{s_3^\ast}{\sqrt{2}} & \frac{-s_1-s_1^\ast+s_2-s_2^\ast}{2} & \frac{-s_2-s_2^\ast+s_1-s_1^\ast}{2} \\
  \frac{s_3^\ast}{\sqrt{2}} & -\frac{s_3^\ast}{\sqrt{2}} & \frac{s_2+s_2^\ast+s_1-s_1^\ast}{2} & -\frac{s_1+s_1^\ast+s_2-s_2^\ast}{2}
\end{bmatrix} \]
Note that this code has unequal transmit power among the antennas, meaning the signal envelope is not constant and the antennas must vary their power each time slot, which is undesirable. An improved version with equal transmit power among the transmitting antennas is~\cite{ganesan01}
\[ \bm{G} = \begin{bmatrix}
  s_1 & s_2 & s_3 & 0 \\
  -s_2^\ast & s_1^\ast & 0 & s_3 \\
  s_3^\ast & 0 & -s_1^\ast & s_2 \\
  0 & s_3^\ast & -s_2^\ast & -s_1
\end{bmatrix}. \]
While this code is orthogonal like the previous example and has a higher rate, it is widely linear rather than linear because there is a correlation between the transmitted symbols and their complex conjugates. Thus, this code trades increased bandwidth efficiency for increased decoding complexity.

\subsubsection{$C(4,4,4)$}
To achieve full-rate for $N_T=4$ transmit antennas, it is necessary to use a QOSTBC. One version described in~\cite{jafarkhani05} has the transmission matrix
\[ \bm{G} = \begin{bmatrix}
  s_1 & s_2 & s_3 & s_4 \\
  -s_2^\ast & s_1^\ast & -s_4^\ast & s_3^\ast \\
  s_3 & s_4 & s_1 & s_2 \\
  -s_4^\ast & s_3^\ast & -s_2^\ast & s_1^\ast
\end{bmatrix}. \]
This code has full-rate but only second-order diversity. In addition, it is linearly decodable.

\subsubsection{$C(8,8,16)$}
For $N_T=8$ transmit antennas, the authors only know of $R=1/2$ codes. The first, given in~\cite{tarokh99}, has the transmission matrix
\[ \bm{G} = \begin{bmatrix*}[r]
  s_1 & s_2 & s_3 & s_4 & s_5 & s_6 & s_7 & s_8 \\
  -s_2 & s_1 & s_4 & -s_3 & s_6 & -s_5 & -s_8 & s_7 \\
  -s_3 & -s_4 & s_1 & s_2 & s_7 & s_8 & -s_5 & -s_6 \\
  -s_4 & s_3 & -s_2 & s_1 & s_8 & -s_7 & s_6 & -s_5 \\
  -s_5 & -s_6 & -s_7 & -s_8 & s_1 & s_2 & s_3 & s_4 \\
  -s_6 & s_5 & -s_8 & s_7 & -s_2 & s_1 & -s_4 & s_3 \\
  -s_7 & s_8 & s_5 & -s_6 & -s_3 & s_4 & s_1 & -s_2 \\
  -s_8 & -s_7 & s_6 & s_5 & -s_4 & -s_3 & s_2 & s_1 \\
  s_1^\ast & s_2^\ast & s_3^\ast & s_4^\ast & s_5^\ast & s_6^\ast & s_7^\ast & s_8^\ast \\
  -s_2^\ast & s_1^\ast & s_4^\ast & -s_3^\ast & s_6^\ast & -s_5^\ast & -s_8^\ast & s_7^\ast \\
  -s_3^\ast & -s_4^\ast & s_1^\ast & s_2^\ast & s_7^\ast & s_8^\ast & -s_5^\ast & -s_6^\ast \\
  -s_4^\ast & s_3^\ast & -s_2^\ast & s_1^\ast & s_8^\ast & -s_7^\ast & s_6^\ast & -s_5^\ast \\
  -s_5^\ast & -s_6^\ast & -s_7^\ast & -s_8^\ast & s_1^\ast & s_2^\ast & s_3^\ast & s_4^\ast \\
  -s_6^\ast & s_5^\ast & -s_8^\ast & s_7^\ast & -s_2^\ast & s_1^\ast & -s_4^\ast & s_3^\ast \\
  -s_7^\ast & s_8^\ast & s_5^\ast & -s_6^\ast & -s_3^\ast & s_4^\ast & s_1^\ast & -s_2^\ast \\
  -s_8^\ast & -s_7^\ast & s_6^\ast & s_5^\ast & -s_4^\ast & -s_3^\ast & s_2^\ast & s_1^\ast
\end{bmatrix*} \]

\subsubsection{$C(8,4,8)$}
Another code for $N_T=8$ is~\cite{larsson03}
\[ \bm{G} = \begin{bmatrix}
  s_1 & 0 & 0 & 0 & -s_4^\ast & 0 & -s_2^\ast & s_3^\ast \\
  0 & s_1 & 0 & 0 & 0 & -s_4^\ast & -s_3 & -s_2 \\
  0 & 0 & s_1 & 0 & s_2 & s_3^\ast & -s_4 & 0 \\
  0 & 0 & 0 & s_1 & -s_3 & s_2^\ast & 0 & -s_4 \\
  s_4 & 0 & -s_2^\ast & s_3^\ast & s_1^\ast & 0 & 0 & 0 \\
  0 & s_4 & -s_3 & -s_2 & 0 & s_1^\ast & 0 & 0 \\
  s_2 & s_3^\ast & s_4^\ast & 0 & 0 & 0 &s_1^\ast & 0 \\
  -s_3 & s_2^\ast & 0 & s_4^\ast & 0 & 0 & 0 & s_1^\ast
\end{bmatrix} \]
While the previous two codes both have the same rate, it is expected that the $C(8,8,16)$ code will perform better since its transmission matrix is not sparse like that of $C(8,4,8)$ code, a greater diversity is expected.
